\documentclass[a4paper,10pt]{article}

\usepackage[right=0.5cm,left=0.5cm,top=0.5cm,bottom=1.75cm]{geometry}
\usepackage{pgf}
\usepackage{tikz}
\usepackage[utf8]{inputenc}
\usepackage{xargs}
\usetikzlibrary{arrows,automata}
\usetikzlibrary{matrix}
\usetikzlibrary{positioning}
\usetikzlibrary{shapes.geometric}
\usetikzlibrary{shapes.multipart}

\tikzset{
	entity/.style={
		rectangle split,
		rectangle split parts=2,
		draw=black, very thick,
		inner sep=2pt,
		text centered,
	},
	table/.style={
		rectangle split,
		rectangle split parts=3,
		draw=black, very thick,
		inner sep=2pt,
		text centered,
	},
	relation/.style={
	ellipse,
	draw=black, very thick,
	inner sep=2pt,
	text centered,
	},
	lien/.style={
	fill=white,
	%above,
	%sloped,
	},
	relationWithAttribut/.style={
		ellipse split,
		draw=black, very thick,
		inner sep=2pt,
		text centered,
	},
}
\newcommandx{\entity}[3][3=]{
\node[entity,#3] (#1) {
	#1
	\nodepart{two}
	\begin{tabular}{l}
		#2
	\end{tabular}};
}
\newcommandx{\relation}[2][2=]{
	\node[relation,#2](#1){#1};
}
\newcommandx{\relationWithAttribut}[3][3=]{
\node[relationWithAttribut,#3](#1)
{#1 \nodepart{lower} #2};
}
\newcommandx{\relationWithAttributs}[3][3=]{
	\node[relationWithAttribut,#3](#1)
	{#1 \nodepart{lower} 	\begin{tabular}{l}
			#2
	\end{tabular}};
}

\newcommand{\lien}[3]{\path (#1) edge node[lien] {#3} (#2);}
\newcommandx{\tableEntity}[4][4=]{
	\node[table,#4] (#1) {
		#1
		\nodepart{two}
		\begin{tabular}{l@{\ \ :}l@{\ \ [}l@{](}l@{)}l  }
			id & uint &id&ID&\\
			#2
		\end{tabular}
		\nodepart{three}
		\begin{tabular}{l}
			primary key (id)\\
			#3
		\end{tabular}
	};
}
\newcommand{\lkey}[2]{\path[<-] (#1) edge (#2);}
\newcommand{\lkeyloop}[1]{\draw[->] (#1.north west) ..controls+(60:1.25cm) and +(120:1.25cm) .. (#1.95);}
\newcommand{\fkey}[2]{foreign key (#1) $\to$ #2}
\newcommand{\unique}[1]{unique (#1)}
%Arbre
\newcommandx{\arbreERD}[2][2=]{
	\node[relation,#2](filiation#1){filiation#1};
	\entity{Arbre#1}{feuille\\num}[above=0.75cm of filiation#1]
	\node[relation,above=1.65cm of Arbre#1](parent#1){parent#1};
}
\newcommand{\lienArbreERD}[1]{
	\lien{Arbre#1.north west}{parent#1.west}{\begin{tabular}{c}fils\\$0-N$\end{tabular}}
	\lien{Arbre#1.north east}{parent#1.east}{\begin{tabular}{c}parent\\$0-1$\end{tabular}}
	\lien{Arbre#1.south}{filiation#1.north}{$1$}
}
\newcommandx{\arbreTable}[2][2=]{
	\tableEntity{Arbre#1}{
		feuille & bool &bool& fl &\\
		num& int &intSup0& num&\\
		parent & uint &idSup1Null& pt & NULL\\
	}{
		\fkey{parent}{Arbre#1}\\
		\unique{parent, num}
	}{#2}
}
\newcommand{\lkeyArbre}[1]{\lkeyloop{Arbre#1}}
\newcommand{\ent}{purple}
\newcommand{\bdd}{blue}
\newcommand{\gui}{violet}
\begin{document}

\section{ERD}
\begin{tikzpicture}[thick]
\footnotesize

	\entity{Prof}{
		nom
	}
	
	\relation{enseigne}[below=0.75cm of Prof]
	
	\entity{Matiere}{
		num\\
		nom
	}[below=0.75cm of enseigne]
	
	\relation{contrainte}[below=1cm of Matiere]
	
	\entity{Horaire}{
		num\\
		code
	}[below=3cm of Matiere]

	\relation{ventilation}[right=2cm of Matiere]
	
	\entity{Creneau}{
		num\\
		code
	}[right=2cm of ventilation]

	\relation{affectation}[below=2cm of ventilation]
	
\lien{Prof}{enseigne}{$0-N$}
\lien{enseigne}{Matiere}{$1$}
\lien{Matiere}{contrainte}{$0-N$}
\lien{contrainte}{Horaire}{$1$}
\lien{Matiere}{ventilation}{$0-N$}
\lien{ventilation}{Creneau}{$1$}
\lien{Creneau}{affectation}{$0-N$}
\lien{affectation}{Horaire}{$0-1$}

\end{tikzpicture}
\subsection{Tables}
\begin{tikzpicture}[thick]
\scriptsize

	\tableEntity{Prof}{
		nom&text&stringNotEmpty&nm&
	}{
		\unique{nom}\\
	}[]

	\tableEntity{Matiere}{
		id\_prof&uint&idSup1&idp\\
		num&int&intSup0&num\\
		nom&text&stringNotEmpty&nm&
	}{
		\fkey{id\_prof}{Prof}\\
		\unique{id\_prof, num}
	}[below=2cm of Prof]
	
	\tableEntity{Horaire}{
		id\_matiere&uint&idSup1&idm\\
		id\_creneau&uint&idSup1Null&idc&NULL\\
		num&int&intSup0&num\\
		code&uint&code&cd
	}{
		\fkey{id\_matiere}{Matiere}\\
		\fkey{id\_creneau}{Creneau}\\
		\unique{id\_matiere, num}
	}[below=2cm of Matiere]
	
	\tableEntity{Creneau}{
		id\_matiere&unit&idSup1&idm\\
		num&int&intSup0&num\\
		code&uint&code&cd
	}{
		\unique{id\_matiere, num}
	}[right=3cm of Matiere]
	

\lkey{Prof}{Matiere}
\lkey{Matiere}{Horaire}
\lkey{Matiere}{Creneau}
\lkey{Creneau}{Horaire}
\end{tikzpicture}	
\end{document}